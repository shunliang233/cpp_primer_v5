\section{类的定义}
\begin{itemize}
	\item 类的定义分三个阶段
	\begin{itemize}
		\item 仅声明类,此时可以定义指向该类的指针和引用,也可以声明(但不能定义)以该类为参数或返回类型的函数。
		\item 定义类,此时类的数据成员已经全部可用了,在类内定义的成员函数也是可用的,只有不在类内定义的成员函数还不能用。
		\item 在类外部定义剩余成员函数,此时该类就拥有了全部功能。
	\end{itemize}
	\item 头文件不能相互包含,因此两个相互调用的类的定义必须放到同一个头文件中。
	\item 按照阶段对两个相互关联的类进行定义,先把两个类的第一阶段定义完,再把两个类的第二阶段定义完,最后把两个类的第三阶段定义完。
	
	
\end{itemize}
